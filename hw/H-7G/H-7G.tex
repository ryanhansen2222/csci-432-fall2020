\documentclass{article}
\usepackage{../fasy-hw}
\usepackage{ wasysym }

%% UPDATE these variables:
\renewcommand{\hwnum}{7}
\title{Advanced Algorithms, Homework \hwnum}
\author{TODO-Put Your Name Here}
\collab{n/a}
\date{due: 9 November 2020}

\begin{document}

\maketitle

This homework assignment should be
submitted as a single PDF file to to Gradescope.

General homework expectations:
\begin{itemize}
    \item Homework should be typeset using LaTex.
    \item Answers should be in complete sentences and proofread.
    \item This HW can be submitted as an individual or as a group.
\end{itemize}

In any question that you are expected to provide an algorithm, you are
expected to provide:
\begin{enumerate}
    \item Describe the problem in your own words, including
        describing what the input and output is.
    \item Describe, in paragraph form, the algorithm you propose.
    \item Provide a nicely formatted algorithm to solve the problem.
    \item Use a decrementing function to prove that algorithm terminates.
            OR  Give the runtime with justification.
    \item Prove partial correctness.  In other words, if there is a loop or
        recursion, what is the loop/recursion invariant? Provide the proof.
        (Note: you only need to do this for the outer-most loop if there are
        nested loops).
\end{enumerate}

\nextprob
\collab{TODO}

Chapter 7, Question 4, Part(a) (Maximum Weight Spanning Tree)

\paragraph{Answer}

% ============================================

TODO: your answer goes between these lines

% ============================================


\nextprob
\collab{TODO}

Chapter 8, Question 4, Part(a) (Removing an Edge)

\paragraph{Answer}

% ============================================

TODO: your answer goes between these lines

% ============================================

\nextprob
\collab{TODO}

Chapter 10, Question 1, (Feasible Flow)

\paragraph{Answer}

% ============================================

TODO: your answer goes between these lines

% ============================================


\nextprob
\collab{TODO}

Chapter 10, Question 4, (Opposing Edges)

\paragraph{Answer}

% ============================================

TODO: your answer goes between these lines

% ============================================

\nextprob
\collab{TODO}

Chapter 11, Question 6, (Mini-Golf)

\paragraph{Answer}

% ============================================

TODO: your answer goes between these lines

% ============================================

\nextprob
\collab{TODO}

Find an algorithm discussed in a recent news article (over the past 12 months).
Choose ONE of the following:
\begin{enumerate}
    \item Look up the primary resource for this algorithm (likely to be a
        research paper).  Compare/contrast the similarities and differences between the
        way the news article describes the problem and algorithm with the way
        that the primary resource describes it.
    \item If the algorithm itself is not given in the article, provide a prose
        description of the algorithm along with pseudocode. (This might require
        looking up the primary resource for the algorithm).
    \item Analyze the runtime of the algorithm.
    \item Prove the correctness of the algorithm.
\end{enumerate}

\paragraph{Answer}

% ============================================

TODO: your answer goes between these lines

% ============================================


\nextprob
\collab{TODO}

Choose an algorithm that you analyzed on a homework in this class (can be this
HW or a previous one).  Suppose you are a journalist writing about this
break-through algorithm and write a one-page summary of the algorithm for a
general audience.  Describing the problem that this algorithm solves and the
applications of the problem should be highlighted (feel free to do some
research).  Detail of the algorithm and proofs of correctness or runtime should
be only given at a very high level.

\paragraph{Answer}

% ============================================

TODO: your answer goes between these lines

% ============================================





\end{document}
