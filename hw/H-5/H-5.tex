\documentclass{article}
\usepackage{../fasy-hw}
\usepackage{ wasysym }

%% UPDATE these variables:
\renewcommand{\hwnum}{5}
\title{Advanced Algorithms, Homework \hwnum}
\author{TODO-Put Your Name Here}
\collab{n/a}
\date{due: 15 October 2020}

\begin{document}

\maketitle

This homework assignment should be
submitted as a single PDF file to to Gradescope.

General homework expectations:
\begin{itemize}
    \item Homework should be typeset using LaTex.
    \item Answers should be in complete sentences and proofread.
\end{itemize}

\nextprob
\collab{TODO}

You should make at least ten contributions to the Piazza board
discussing the solutions to Problems in Chapter 5 or 6 of the textbook.  Your
contribution does not have to be a complete solution.  It can be any element of
a full solution to a problem requiring an algorithm as an answer.  (For this
question, the outcomes are: insufficient posts (-1), low pass (+1), pass (+3),
and high pass (+5).

Caveat: The number of contributions that can count from a single day are the
number of days before the assignment is due +1.  (So, Thursday the 15th, you can
have one post count, the day before two, etc.)

\paragraph{Answer}

% ============================================

My contributions are:
\begin{enumerate}
    \item (TODO: state the problem number, and date/time). TODO:
        copy the post here.
    \item ...
\end{enumerate}

% ============================================

\nextprob
\collab{TODO}

Chapter 4, Question 1 (Greedy Schedule).  I encourage you to think through all 9
alternative schedules.  However, you only need to hand in two:
\begin{enumerate}
    \item Choose one alternate strategy that
        works, and prove that it works.
    \item Choose one alternate strategy that does not work, and give a
        counter-example.
\end{enumerate}

\paragraph{Answer}
% ============================================

TODO: your answer goes between these lines

% ============================================


\nextprob
\collab{TODO}

Chapter 4, Question 3 (Interval Covering).

This question asks you to come up with an algorithm.  As a reminder, you are
expected to provide:
\begin{enumerate}
    \item Describe the problem in your own words, including
        describing what the input and output is.
    \item Describe, in paragraph form, the algorithm you propose.
    \item Provide a nicely formatted algorithm to solve the problem.
    \item Use a decrementing function to prove that algorithm terminates.
            OR  Give the runtime with justification.
    \item Prove partial correctness.  In other words, if there is a loop or
        recursion, what is the loop/recursion invariant? Provide the proof.
        (Note: you only need to do this for the outer-most loop if there are
        nested loops).
\end{enumerate}



\paragraph{Answer}
% ============================================

TODO: your answer goes between these lines

% ============================================

\nextprob
\collab{TODO}

Suppose someone poses a problem to you, and you have a hunch that it can be
solved with a dynamic program.  Describe, in your own words, the steps you will
take to work through finding a solution to the problem.  If it helps, you can
choose an example to illustrate working through the process.

\paragraph{Answer}

% ============================================

TODO: your answer goes between these lines

% ============================================



\nextprob
\collab{TODO}

Choose one concept or algorithm that you have learned
in this class so far. Describe it to someone who has
taken 232 and 246, but not 432.

\paragraph{Answer}

% ============================================

TODO: your answer goes between these lines

% ============================================

\nextprob
\collab{TODO}

The final course project is to research a ``recent'' video and make a
five-minute video about it.  This semester, I am offering an alternate
assignment: to give complete (polished) solutions to five algorithms (of a
choice of 10).
Please state your preference among the following:

\begin{enumerate}
    \item You would like to do the standard project, and you know who you want
        to work with (let us know here).
    \item You would like to do the standard project, but you would like us to
        assign your group.
    \item You would like to do the alternate assignment instead.
\end{enumerate}

\paragraph{Answer}

% ============================================

TODO: your answer goes between these lines

% ============================================



\end{document}
