\documentclass{article}
\usepackage{../fasy-hw}
\usepackage{ wasysym }

%% UPDATE these variables:
\renewcommand{\hwnum}{5}
\title{Advanced Algorithms, Homework \hwnum}
\author{TODO-Put Your Name Here}
\collab{n/a}
\date{due: 15 October 2020}

\begin{document}

\maketitle

This homework assignment should be
submitted as a single PDF file to to Gradescope.

General homework expectations:
\begin{itemize}
    \item Homework should be typeset using LaTex.
    \item Answers should be in complete sentences and proofread.
\end{itemize}

\nextprob
\collab{TODO}

You should make at least ten contributions to the Piazza board
discussing the solutions to Problems in Chapter 5 or 6 of the textbook.  Your
contribution does not have to be a complete solution.  It can be any element of
a full solution to a problem requiring an algorithm as an answer.  (For this
question, the outcomes are: insufficient posts (-1), low pass (+1), pass (+3),
and high pass (+5).

Caveat: The number of contributions that can count from a single day are the
number of days before the assignment is due +1.  (So, Thursday the 15th, you can
have one post count, the day before two, etc.)

\paragraph{Answer}

% ============================================

My contributions are:
\begin{enumerate}
    \item (TODO: state the problem number, and date/time). TODO:
        copy the post here.
    \item ...
    \item Yeah started too late not gonna do that
\end{enumerate}

% ============================================

\nextprob
\collab{TODO}

Chapter 4, Question 1 (Greedy Schedule).  I encourage you to think through all 9
alternative schedules.  However, you only need to hand in two:
\begin{enumerate}
    \item Choose one alternate strategy that
        works, and prove that it works.

		Option i)
		Suppose we have an optimal solution that differs from our
		greedy algorithm.  Suppose
		we could exchange the option in our solution $x$ with two or more
		courses (in the optimal), $y,z$. Since they are replacing our single course,
		they must both conflict with $x$. We know our course contains no courses and ends last, meaning
		$y,z$ starts before $x$. Thus $y,z$ conflict with eachother as well. Then the size of our solution is equal to the size of the optimal solution.
		
    \item Choose one alternate strategy that does not work, and give a
        counter-example.

	Option a) choosing the course x that ends last and discarding classes
		that conflict is incorrect. The easy counterexample is if x
		is super long and every other class is contained within x.
\end{enumerate}

\paragraph{Answer}
% ============================================

TODO: your answer goes between these lines

% ============================================


\nextprob
\collab{TODO}

Chapter 4, Question 3 (Interval Covering).

This question asks you to come up with an algorithm.  As a reminder, you are
expected to provide:
\begin{enumerate}
    \item Describe the problem in your own words, including
        describing what the input and output is.

		Input is $X$, the set of intervals on the real line. Our output
		is the minimal cover of $X$, $Y\subset X$, meaning the smallest subset of $X$
		such that $\Cup_{x\in X} = \Cup_{y \in Y}$
    \item Describe, in paragraph form, the algorithm you propose.
	    Build the solution from left to right greedily. Take the intervals
		that end furthest right so long as their leftmost edge is left
		enough.
    \item Provide a nicely formatted algorithm to solve the problem.
\begin{algorithm}[H]
 \KwData{Big Set, $X$}
 \KwResult{Cover, $Y$}\\
	$Y$ = 0\\

	rightmost = argmax($X.right$)\\
	leftmost = argmin($X.left$)\\


	\While{leftmost.right $<$ rightmost.right}
	{

		Y.append(leftmost)\\
		X.delete(leftmost)\\

		\For{interval in $X$}
		{
			\If{interval.left $<$ leftmost.right or interval.left == min($X.left$}\\{leftmost = interval}
			}
			}

 \caption{GreedyIntervalCover}
\end{algorithm}
    \item Use a decrementing function to prove that algorithm terminates.
            OR  Give the runtime with justification.
		$n^2$. We add one element of optimal value to our minimal set
		each cycle, and there can be potentially every value in $Y$, so
		we have $n$ number of potential rotations. Second, we loop through each element in $X$ each cycle, meaning we run in $O(n^2)$ time.
		
    \item Prove partial correctness.  In other words, if there is a loop or
        recursion, what is the loop/recursion invariant? Provide the proof.
        (Note: you only need to do this for the outer-most loop if there are
        nested loops).

		The outer loop here is the while loop. Its goal is to return
		the minimal cover of $X$. The loop guard checks to see if we
		reached the rightmost edge of $X$, meaning we finish. The loop
		invariant is as follows - Y contains only elements of $X$ in 
		its minimal cover, and the elements it does contain are the 
		minimal cover from the leftmost edge of $X$ to the rightmost 
		edge of $Y$ as a subset of $X$. 

		Base case: $Y$ starts as empty, so it is trivially true.\\
		Inductive step: Suppose $Y$ only contains elements in $X$s minimal cover. Since we know the minimal cover must contain every point in $X$, we know one of two things must happen. One, if the minimal cover is disjoint at the current boundary, we need the new leftmost unattained edge. Otherwise, we need a set that intersects the boundary. Otherwise, we will fail to have a cover. Then
		we greedily search for sets that satisfy this property of largest rightmost edge. The greedy approach is justified from the exchange principle - If we exchanged the largest set with a smaller set in the optimal solution, we would find its unique data is a subset of the former, meaning
		the modified $Y$  must indeed be a part of the left partial minimal cover as
		described above. 
\end{enumerate}



\paragraph{Answer}
% ============================================

TODO: your answer goes between these lines

% ============================================

\nextprob
\collab{TODO}

Suppose someone poses a problem to you, and you have a hunch that it can be
solved with a dynamic program.  Describe, in your own words, the steps you will
take to work through finding a solution to the problem.  If it helps, you can
choose an example to illustrate working through the process.

\paragraph{Answer}

% ============================================

I did this super recently two times! So there is this google challenge called
the foobar Ben sent me a link to. Two of its problems I solved with dynamic 
programming. I thought about if a given of those problems had steps that build on each other. If it had some kind of ''generations" aspect where results depend on previous
generations, I would start thinking dynamic programming. If I answer yes to the aforementioned question, I try to frame the
problem either recursively or in the breadth first search queue implementation.
% ============================================



\nextprob
\collab{TODO}

Choose one concept or algorithm that you have learned
in this class so far. Describe it to someone who has
taken 232 and 246, but not 432.

\paragraph{Answer}

% ============================================

Loop invariants and proving correctness of an algorithm. As a man of theory, I
often think of methodologies for how to solve problems. Many of them work for
most of the canonical cases, but there are occasionally edge cases that are super easy to miss. Proofs of correctness can do a good job to reveal weaknesses in your 
implementations/reasoning and help guarantee you are doing what you want, which
is awesome. Loop invariants are particularly interesting because often times
you know what you want the loop to do, but stating precisely what it is doing
is kind of tough. Being able to frame what the goal is, what is changing, and 
how the change makes progress to your end goal is super cool and useful in 
hashing out good code and problem solving!

% ============================================

\nextprob
\collab{TODO}

The final course project is to research a ``recent'' video and make a
five-minute video about it.  This semester, I am offering an alternate
assignment: to give complete (polished) solutions to five algorithms (of a
choice of 10).
Please state your preference among the following:

\begin{enumerate}
    \item You would like to do the standard project, and you know who you want
        to work with (let us know here).
		Yes I want to work with Ben and Seth (I havent talked to them,
		but they are great). The 5 algorithm solutions seems boring.
    \item You would like to do the standard project, but you would like us to
        assign your group.
    \item You would like to do the alternate assignment instead.
\end{enumerate}

\paragraph{Answer}

% ============================================

TODO: your answer goes between these lines

% ============================================



\end{document}
