\documentclass{article}
\usepackage{../fasy-hw}
\usepackage{ wasysym }

%% UPDATE these variables:
\renewcommand{\hwnum}{1}
\title{Advanced Algorithms, Homework 1}
\author{TODO-Your Name Here}
\collab{n/a}
\date{due: 27 August 2020}

\begin{document}

\maketitle

This homework assignment is due on 27 August 2020, and should be
submitted as a single PDF file to D2L and to Gradescope.

General homework expectations:
\begin{itemize}
    \item Homework should be typeset using LaTex.
    \item Answers should be in complete sentences and proofread.
\end{itemize}

\nextprob
\collab{n/a}

Answer the following questions:
\begin{enumerate}
    \item What is your elevator pitch?  Describe yourself in 1-2
                sentences.
    \item What was your favorite CS class so far, and why?
    \item What was your least favorite CS class so far, and why?
    \item Why are you interested in taking this course?
    \item What is your biggest academic or research goal for this semester (can
        be related to this course or not)?
    \item What do you want to do after you graduate?
    \item What was the most challenging aspect of your coursework last semester
        after the university transitioned to online?
    \item What went well last semester for you after the university transitioned
        to online?
\end{enumerate}

\paragraph{Answer}

% ============================================

TODO: your answer goes between these lines

% ============================================

\nextprob
\collab{n/a}

Please do the following:
\begin{enumerate}
    \item Write this homework in LaTex.
        Note: if you have not used LaTex before and this is an
        issue for you, please contact the instructor or TA.
    \item Update your photo on D2L to be a recognizable headshot of you.
    \item Sign up for the class discussion board.
\end{enumerate}

\paragraph{Answer}

% ============================================

TODO: write a statement confirming you have completed these tasks.

% ============================================


\nextprob
\collab{}

    In this class,
    please properly cite all resources that you use.
    To refresh your memory on what plagiarism is,
    please
    complete the plagiarism tutorial found here:
    \url{http://www.lib.usm.edu/plagiarism_tutorial}.
    If you have observed plagiarism or cheating in a classroom (either as an
    instructor or as a student), explain the situation and how it made you
    feel.  If you have not experienced plagiarism or cheating or if you would
    prefer not to reflect on a personal experience, find a news
    article about plagiarism or cheating and explain how you would feel if you
    were one of the people involved.


\paragraph{Answer}

% ============================================

TODO: your answer goes between these lines

% ============================================



\nextprob
Prove the following statement: Every tree with one or more nodes/vertices has
exactly $n-1$ edges.

\paragraph{Answer}

% ============================================

TODO: your answer goes between these lines

% ============================================



\nextprob
Use the definition of big-O notation to prove that $f(x)=n^2 + 3n +2$ is
$O(n^2)$.

\paragraph{Answer}

% ============================================

TODO: your answer goes between these lines

% ============================================



\nextprob
Consider the \textsc{RightAngle} algorithm on page 8 of the textbook.
\begin{enumerate}
    \item When we design an algorithm, we design the algorithm to solve a
        problem or answer a question.  What is the problem that this algorithm
        solves?
    \item Prove that the algorithm terminates.
\end{enumerate}

\paragraph{Answer}

% ============================================

TODO: your answer goes between these lines

% ============================================



\nextprob
Consider the following statement: If $a$ and $b$ are both even numbers, then $ab$ is
an even number.
\begin{enumerate}
    \item What is the definition of an odd number?
    \item What is the definition of an even number?
    \item What is the contrapositive of this statement?
    \item What is the converse of this statement?
    \item Prove this statement.
\end{enumerate}

\paragraph{Answer}

% ============================================

TODO: your answer goes between these lines.  Be sure to enumerate!

% ============================================



\end{document}

